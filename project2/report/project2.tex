\documentclass[11pt]{article}

\usepackage[letterpaper]{geometry}
%\usepackage[letterpaper,left=2.5cm,top=2cm,right=2.5cm,bottom=2cm]{geometry}

\usepackage[utf8]{inputenc}
\usepackage{mathpazo}
\usepackage{amsmath}
\usepackage{amsfonts}
\usepackage{siunitx}
\usepackage{cancel}

\usepackage{graphicx}
\usepackage{float}
\usepackage{empheq}
\usepackage[most]{tcolorbox}

\newtcbox{\mymath}[1][]{%
  nobeforeafter, math upper, tcbox raise base,
  enhanced, colframe=black!30!black,
  colback=yellow!30, boxrule=1pt,
  #1}

% Hyperlinks with decent looking default colors.
\usepackage{hyperref}
\usepackage{xcolor}
\hypersetup{
  colorlinks,
  linkcolor={red!50!black},
  citecolor={blue!50!black},
  urlcolor={blue!80!black}
}

% For those sexy spaced low small caps from classic-thesis!
\usepackage{microtype}
\usepackage{textcase}
\DeclareRobustCommand{\spacedlowsmallcaps}[1]{%
  \textls[80]{\scshape\MakeTextLowercase{#1}}%
}

\usepackage{fancyhdr}
\pagestyle{fancy} 
\fancyhead{}
\rhead{Ali Ramadhan}
\lhead{6.339 Project 2---Finite Volume Methods}
\cfoot{\thepage}

\title{\spacedlowsmallcaps{6.339: Numerical Methods for Partial Differential Equations}\\ \spacedlowsmallcaps{Project two: Finite Volume Methods}}
\author{Ali Ramadhan$^\text{†}$ (\texttt{alir@mit.edu})}
\date{\textit{$^\text{†}$Department of Earth, Atmospheric, and Planetary Sciences}}

\renewcommand\thesection{\Alph{section}}

\begin{document}
\maketitle

In this project, we will utilize finite volume methods to study dense traffic flow and traffic jams modeled as shockwaves. We model traffic in each lane by a scalar hyperbolic conservation law, following what is known as the Lighthill-Whitman-Richards model.

We use a scalar hyperbolic conservation law to model traffic density $\rho^{(\ell)}(x,t)$ for $n$ lanes indexed by $\ell = 1,2,\dots,n$
\begin{equation}
	\frac{\partial \rho^{(\ell)}}{\partial t} + \frac{\partial (\rho^{(\ell)} v^{(\ell)})}{\partial x} = s
\end{equation}
where $v^{(\ell)}(x,t)$ is the average velocity of the cars. This, however, provides us with only one equation for two unknowns and thus we specify the velocity by
\begin{equation}
	v(\rho) = v_\mathrm{max} \left( 1 - \frac{\rho^2}{\rho_\mathrm{max}^2} \right)
\end{equation}
giving us a traffic flux of 
\begin{equation}
f(\rho) = \rho v = v_\mathrm{max} \left( \rho - \frac{\rho^3}{\rho_\mathrm{max}^2} \right)
\end{equation}
The source term
\begin{equation}
	s^{(\ell)} = \alpha \sum_{\substack{|k-\ell|=1 \\ 1 \le k,\ell \le n}} \rho^{(k)} - \rho^{(\ell)}
\end{equation}
models the density of traffic that is switching lanes from neighboring lanes. $\alpha$ is the fraction of drivers that change lanes.

We will split up our one-dimensional grid into a number of cells indexed by $i=1,2,\ldots,N$. We will index the edges of the cell $i$ by $i-\frac{1}{2}$ for the left boundary of the cell, and by $i+\frac{1}{2}$ for the right boundary of the cell. So we can think of $i$ as indexing the cell centers.

To derive a first-order conservative finite-volume scheme for a single lane, we will consider the volume averages of the traffic density $\rho(x,t)$ at two different times. The volume average of the traffic density at cell $i$, $\rho_i = \rho(x_i,t)$, at a time $t_1$ over $x \in \left[ x_{i-\frac{1}{2}}, x_{i+\frac{1}{2}} \right]$ must exist by the mean value theorem and is given by
\begin{equation*}
	\bar{\rho}_i(t_1) = \frac{1}{\Delta x_i} \int_{x_{i-\frac{1}{2}}}^{x_{i+\frac{1}{2}}} \rho(x,t_1) \; dx
\end{equation*}
and an identical expression can be written for the volume average at a later time $t_2 > t_1$. Now, integrating the scalar conservation law in time from $t = t_1$ to $t = t_2$ we can write
\begin{equation*}
	\int_{t_1}^{t_2} \frac{\partial\bar{\rho}}{\partial t} \; dt + \int_{t_1}^{t_2} \frac{\partial(\bar{\rho} v)}{\partial x} \; dt = 0
\end{equation*}                         
where the first integral can be evaluated using the second fundamental theorem of calculus, sometimes referred to as the Newton–Leibniz axiom, and rearranged to obtain $\bar{\rho}_i$ at a later time
\begin{equation*}
\bar{\rho}(x,t_2) = \bar{\rho}(x,t_1) - \int_{t_1}^{t_2} \frac{\partial(\bar{\rho}_i v_i)}{\partial x} \; dt
\end{equation*}

We can now calculate $\rho_i(t_2)$ as
\begin{align*}
\bar{\rho}_i(t_2) &=
	\frac{1}{\Delta x_i} \int_{x_{i-\frac{1}{2}}}^{x_{i+\frac{1}{2}}} \left[ \rho(x,t_1) - \int_{t_1}^{t_2} \frac{\partial(\rho v)}{\partial x} \; dt \right] \; dx \\
	&= \frac{1}{\Delta x_i} \int_{x_{i-\frac{1}{2}}}^{x_{i+\frac{1}{2}}} \rho(x,t_1) \; dx - \frac{1}{\Delta x_i} \int_{x_{i-\frac{1}{2}}}^{x_{i+\frac{1}{2}}} \int_{t_1}^{t_2} \frac{\partial(\rho v)}{\partial x} \; dt \; dx \\
	&= \bar{\rho}_i(t_1) - \frac{1}{\Delta x_i} \int_{t_1}^{t_2} \left[ \rho(x_{i+\frac{1}{2}}, t) v(x_{i+\frac{1}{2}}, t) - \rho(x_{i-\frac{1}{2}}, t) v(x_{i-\frac{1}{2}}, t) \right] \; dt \\
	&= \bar{\rho}_i(t_1) - \frac{1}{\Delta x_i} \left[ \int_{t_1}^{t_2} F_{i+\frac{1}{2}} - F_{i-\frac{1}{2}} \right] \; dt
\end{align*}
which can be rearranged to write
\begin{equation*}
	\bar{\rho}_i(t_2) - \bar{\rho}_i(t_1) = \frac{d}{dt} \int_{t_1}^{t_2} \rho_i(t) \; dt = \int_{t_1}^{t_2} \left( - \frac{F_{i+\frac{1}{2}} - F_{i-\frac{1}{2}}}{\Delta x_i} \right) \; dt
\end{equation*}
where the integrands inside the two integrals must be the same so that
\begin{equation*}
	\frac{d \bar{\rho}_i}{dt} = - \frac{F_{i+\frac{1}{2}} - F_{i-\frac{1}{2}}}{\Delta x_i}
\end{equation*}
and if we approximate the time derivate by a first-order forward difference finite difference operator $\dot{\bar{\rho}}_i = (\bar{\rho}_i^{n+1} - \bar{\rho}_i^n)/\Delta t$ and further rearrange, we obtain
\begin{equation}
	\bar{\rho}_i^{n+1} = \bar{\rho}_i^n - \frac{\Delta t}{\Delta x_i} \left( F_{i+\frac{1}{2}} - F_{i-\frac{1}{2}} \right) 
\end{equation}

\end{document}